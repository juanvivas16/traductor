%%%%%%%%%%%%%%%%%%%%%%%%%%%%%%%%%%%%%%%%%
% University/School Laboratory Report
% LaTeX Template
% Version 3.1 (25/3/14)
%
% This template has been downloaded from:
% http://www.LaTeXTemplates.com
%
% Original author:
% Linux and Unix Users Group at Virginia Tech Wiki
% (https://vtluug.org/wiki/Example_LaTeX_chem_lab_report)
%
% License:
% CC BY-NC-SA 3.0 (http://creativecommons.org/licenses/by-nc-sa/3.0/)
%
%%%%%%%%%%%%%%%%%%%%%%%%%%%%%%%%%%%%%%%%%

%----------------------------------------------------------------------------------------
%	PACKAGES AND DOCUMENT CONFIGURATIONS
%----------------------------------------------------------------------------------------

%\documentclass{article}
\documentclass[aspectratio=169]{article}

\usepackage[version=3]{mhchem} % Package for chemical equation typesetting
\usepackage{siunitx} % Provides the \SI{}{} and \si{} command for typesetting SI units
\usepackage{graphicx} % Required for the inclusion of images
\usepackage{natbib} % Required to change bibliography style to APA
\usepackage{amsmath} % Required for some math elements
\usepackage[spanish]{babel}
\usepackage{hyperref}
\usepackage{float}
\usepackage{enumerate}
\usepackage{lmodern}
\usepackage[T1]{fontenc}
\usepackage{mathtools}
\usepackage[usenames,dvipsnames]{xcolor}
\usepackage{lscape}
\usepackage{pdflscape}
\usepackage{fancyvrb}
\usepackage{listings}
\usepackage{fancybox}

\setlength\parindent{0pt} % Removes all indentation from paragraphs

\renewcommand{\labelenumi}{\alph{enumi}.} % Make numbering in the enumerate environment by letter rather than number (e.g. section 6)

%\usepackage{times} % Uncomment to use the Times New Roman font

%----------------------------------------------------------------------------------------
%	DOCUMENT INFORMATION
%----------------------------------------------------------------------------------------

\title{Compiladores U-2017 \\ Informe Proyecto \\ Traductor \\ C a Bash} % Title

\author{Juan Andr\'es Vivas\\
		Juli\'an Brice\~no} % Author name

\date{\today} % Date for the report

\begin{document}

\maketitle % Insert the title, author and date

\begin{center}
\begin{tabular}{l r}

\end{tabular}
\end{center}

\newpage

\section{Descripci\'on del problema}

	Se plantea hacer una traducci\'on simple del lenguaje C a Bash, cabe destacar que se est\'a
	pasando de un lenguaje fuertemente tipado a uno debilmente tipado.\\

	El objetivo es traducir expresiones matem\'aticas sencillas, lectura de variables, impresiones
	por pantalla y la estructura de decision if y de repeticion while.

\subsection{Alcance}

	Lograr traducir expresiones matem\'aticas simples, tales como: sumas, restas, multiplicaciones,
	divisiones y calculo de modulo. Las expresiones se limitaran a tener un solo tipo de operaci\'on. \\

	Se tendran expresiones l\'ogicas las cuales sirven para las estructuras if y while, limitadas a un
	solo operador l\'ogico.\\

	Se leera una sola variable por la entrada estandar y se acoto a 3 varibles la cantidad que se pueden
	imprimir.\\

	El tipo de datos en C a usar son int, float y char. A pesar de reconocer los flotantes, al momento de traducir
	solo se tomara en cuenta su parte entera.

\subsection{Especificaci\'on del lenguaje fuente}

	El lenguaje fuente es C reducido, limitado a operaciones matematicas sencillas y las estructuras
	if y while. Se reconocera las bibliotecas que pueda tener y lo que este dentro de la funcion
	main sin ningun par\'ametro. \\

	Tiene los tipos de datos entero, caracter y flotante. Se podran declarar y asignar variables y valores.\\

	Se leera una sola variable por la entrada estandar y la salida se acoto hasta 3 variables. \\

	Todo lo demas no sera reconocido por nuestra gramatica ni traducido.

\subsection{Especificaci\'on del lenguaje a traducir}

El lenguaje de consola Bash (Bourne again shell) interpreta ordenes en una shell de Unix.
Interpretara operaciones matematicas simples, comparacion de variables, lectura y escritura de variables
por la entrada y salida estandar.

\section{Soluci\'on al problema planteado}

La traduccion se realiza a partir de verificar la correctitud del lenguaje fuente, de lo contrario
no se realiza.\\

Se divide en las siguiente secciones.

\subsection{An\'alisis L\'exico}

Se encarga de detectar todos los lexemas y pasarlos al analisis sintactico como tokens.\\
Contiene el siguiente conjunto de expresiones regulares:\\

\begin{lstlisting}

LETRA [a-zA-Z_]
DIGITO [0-9]
{LETRA}({LETRA}|{DIGITO})*
(-)?{DIGITO}+
(-)?{DIGITO}*"."{DIGITO}+

\end{lstlisting}

Las cuales hacen match para detectar identificadores, tipos de datos, palabras reservadas y
valores numericos.\\

Tambien detecta el siguiente conjunto de caracteres: \\

\begin{lstlisting}
. ; { } ( ) < > ! = + - * / ^ : # &
<= >= != += -= *= /= %= == || &&
++ -- /% , ' "
\end{lstlisting}

Cada expresion regular envia un token al analizador sintactico al igual que cada caracter antes descrito
produce un token.\\

El analizador lexico tambien se encarga de eliminar del lenguaje fuente los espacios en blanco, saltos
de linea y tabulaciones. Cuenta saltos de linea y rechaza aquellos caracteres que no coincidan con ningun
patron valido definido.


\subsection{An\'alisis Sint\'actico}

El analisis sintactico se realiza a traves de la siguiente gramatica, el cual recibe los tokens
y los agrupa de acuerdo a las reglas de produccion de la gramatica que se presenta a continuacion.

\begin{lstlisting}

programa:
	codigo;

codigo:
	cabecera principal | principal;

cabecera:
	cabecera NUMERAL RESERVADA MENOR ID MAYOR
	| NUMERAL RESERVADA MENOR ID MAYOR
	| cabecera NUMERAL RESERVADA COMILLAS TEXTO COMILLAS
	| NUMERAL RESERVADA COMILLAS TEXTO COMILLAS
	| cabecera NUMERAL RESERVADA MENOR ID PUNTO ID MAYOR
	| NUMERAL RESERVADA MENOR ID PUNTO ID MAYOR;

principal:
	TIPO RESERVADA PARENTESISABR PARENTESISCERR LLAVEABR cuerpo LLAVECERR;

cuerpo:
	asignacion cuerpo | asignacion | declaracion cuerpo | declaracion
	| retornar cuerpo | retornar | scan cuerpo | scan
	| print cuerpo | print | estructura cuerpo | estructura
	| RESERVADA LLAVEABR cuerpo LLAVECERR estructura PTOCOMA cuerpo
	| RESERVADA LLAVEABR cuerpo LLAVECERR estructura PTOCOMA
	| estructura LLAVEABR cuerpo LLAVECERR cuerpo
	| estructura LLAVEABR cuerpo LLAVECERR
	| RESERVADA LLAVEABR cuerpo LLAVECERR cuerpo
	| RESERVADA LLAVEABR cuerpo LLAVECERR | RESERVADA cuerpo;

estructura:
	RESERVADA PARENTESISABR condicional PARENTESISCERR;

scan:
	RESERVADA PARENTESISABR COMILLAS PRCVAL COMILLAS COMA
	AMPERSAND ID PARENTESISCERR PTOCOMA;

print:
	RESERVADA PARENTESISABR COMILLAS TEXTO COMILLAS PARENTESISCERR PTOCOMA
	| RESERVADA PARENTESISABR COMILLAS TEXTO PRCVAL TEXTO COMILLAS COMA
	ID PARENTESISCERR PTOCOMA
	| RESERVADA PARENTESISABR COMILLAS PRCVAL TEXTO COMILLAS COMA
	ID PARENTESISCERR PTOCOMA
	| RESERVADA PARENTESISABR COMILLAS TEXTO PRCVAL COMILLAS COMA
	ID PARENTESISCERR PTOCOMA
	| RESERVADA PARENTESISABR COMILLAS PRCVAL COMILLAS COMA ID
	PARENTESISCERR PTOCOMA
	| RESERVADA PARENTESISABR COMILLAS PRCVAL TEXTO PRCVAL COMILLAS
	COMA ID COMA ID PARENTESISCERR PTOCOMA
	| RESERVADA PARENTESISABR COMILLAS PRCVAL TEXTO PRCVAL TEXTO PRCVAL
	COMILLAS COMA ID COMA ID COMA ID PARENTESISCERR PTOCOMA;

condicional:
	ID IGUALD ID | NUM IGUALD ID | ID IGUALD NUM
	| ID MAYOR ID | ID MAYOR_I ID | ID MENOR ID | ID MENOR_I ID
	| NUM MAYOR ID | NUM MAYOR_I ID | NUM MENOR ID | NUM MENOR_I ID
	| ID MAYOR NUM | ID MAYOR_I NUM | ID MENOR NUM | ID MENOR_I NUM;

retornar:
	RESERVADA NUM PTOCOMA | RESERVADA ID PTOCOMA
	| RESERVADA PARENTESISABR NUM PARENTESISCERR PTOCOMA
	| RESERVADA PARENTESISABR ID PARENTESISCERR PTOCOMA;


declaracion:
	TIPO ID PTOCOMA | TIPO ID IGUAL NUM PTOCOMA
	| TIPO ID IGUAL COMISIMPLE ID COMISIMPLE PTOCOMA
	| TIPO ID IGUAL COMISIMPLE NUM COMISIMPLE PTOCOMA
	| TIPO ID IGUAL ID PTOCOMA;

asignacion:
	ID IGUAL ID PTOCOMA | ID IGUAL NUM PTOCOMA | ID SUM_ASSIGN ID PTOCOMA
	| ID SUM_ASSIGN NUM PTOCOMA | ID SUB_ASSIGN ID PTOCOMA
	| ID SUB_ASSIGN NUM PTOCOMA | ID MUL_ASSIGN ID PTOCOMA
	| ID MUL_ASSIGN NUM PTOCOMA | ID IGUAL suma PTOCOMA
	| ID IGUAL resta PTOCOMA | ID IGUAL multi PTOCOMA
	| ID IGUAL div PTOCOMA | ID IGUAL ID PORCENTAJE ID PTOCOMA
	| ID IGUAL ID PORCENTAJE NUM PTOCOMA
	| ID IGUAL NUM PORCENTAJE ID PTOCOMA
	| ID IGUAL NUM PORCENTAJE NUM PTOCOMA | ID INC PTOCOMA
	| ID DEC PTOCOMA | INC ID PTOCOMA | DEC ID PTOCOMA;

suma:
	suma SUMA ID | suma SUMA NUM | ID SUMA ID | NUM SUMA NUM
	| ID SUMA NUM | NUM SUMA ID;

resta:
	resta MENOS ID | resta MENOS NUM | ID MENOS ID | NUM MENOS NUM
	| ID MENOS NUM | NUM MENOS ID;

multi:
	multi MULTI ID | multi MULTI NUM | ID MULTI ID | NUM MULTI NUM
	| ID MULTI NUM | NUM MULTI ID;

div:
	div DIV ID | div DIV NUM | ID DIV ID | NUM DIV NUM
	| ID DIV NUM | NUM DIV ID;

\end{lstlisting}

La tabla de simbolos se implemento con un vector dinamico de tuplas, el cual agrupa el tipo de dato
con su identificador.


\subsection{An\'alisis Sem\'antico}

Se encarga de verificar que las palabras reservadas esten correctamente ubicadas dentro del lenguaje fuente.
Las variables deben ser declaradas antes de ser usadas y al momento de hacer una asignacion los
tipos de datos sean iguales.

\subsection{Manejo de errores}

Empieza en el analisis lexico, lleva la cuenta de las lineas que se van leyendo para poder decir
en que linea se ubica un error en el caso de ser detectado, ademas de emitir un mensaje si no se llega
a reconocer alguna palabra.\\

El analisis semantico falla al momento de que no se pueda construir un arbol de derivacion correcto
a partir de la gramatica definida. Se detiene la traduccion y se avisa en que linea esta el error sintactico.\\

Al momento de revisar la semantica, se valida que todas las variables sean declaradas antes de invocarse,
al momento de ser asignadas tengan el mismo tipo de dato y por ultimo que las palabras reservadas coincidan
con su ubicacion y uso dentro del lenguaje. De lo contrario se detiene la traduccion y se emite un mensaje
de error con la linea donde se encontro y la variable o palabra reservada involucrada.





\end{document}
