%%%%%%%%%%%%%%%%%%%%%%%%%%%%%%%%%%%%%%%%%
% University/School Laboratory Report
% LaTeX Template
% Version 3.1 (25/3/14)
%
% This template has been downloaded from:
% http://www.LaTeXTemplates.com
%
% Original author:
% Linux and Unix Users Group at Virginia Tech Wiki
% (https://vtluug.org/wiki/Example_LaTeX_chem_lab_report)
%
% License:
% CC BY-NC-SA 3.0 (http://creativecommons.org/licenses/by-nc-sa/3.0/)
%
%%%%%%%%%%%%%%%%%%%%%%%%%%%%%%%%%%%%%%%%%

%----------------------------------------------------------------------------------------
%	PACKAGES AND DOCUMENT CONFIGURATIONS
%----------------------------------------------------------------------------------------

%\documentclass{article}
\documentclass[aspectratio=169]{article}

\usepackage[version=3]{mhchem} % Package for chemical equation typesetting
\usepackage{siunitx} % Provides the \SI{}{} and \si{} command for typesetting SI units
\usepackage{graphicx} % Required for the inclusion of images
\usepackage{natbib} % Required to change bibliography style to APA
\usepackage{amsmath} % Required for some math elements
\usepackage[spanish]{babel}
\usepackage{hyperref}
\usepackage{float}
\usepackage{enumerate}
\usepackage{lmodern}
\usepackage[T1]{fontenc}
\usepackage{mathtools}
\usepackage[usenames,dvipsnames]{xcolor}
\usepackage{lscape}
\usepackage{pdflscape}
\usepackage{fancyvrb}
\usepackage{listings}
\usepackage{fancybox}
\usepackage [utf8]{inputenc}

\setlength\parindent{0pt} % Removes all indentation from paragraphs

\renewcommand{\labelenumi}{\alph{enumi}.} % Make numbering in the enumerate environment by letter rather than number (e.g. section 6)

%\usepackage{times} % Uncomment to use the Times New Roman font

%----------------------------------------------------------------------------------------
%	DOCUMENT INFORMATION
%----------------------------------------------------------------------------------------

\title{Compiladores U-2017 \\ Informe Proyecto \\ Traductor \\ C a Bash} % Title

\author{Juan Andr\'es Vivas\\
		Juli\'an Brice\~no} % Author name

\date{\today} % Date for the report

\begin{document}

\maketitle % Insert the title, author and date

\begin{center}
\begin{tabular}{l r}

\end{tabular}
\end{center}

\newpage

\section{Descripci\'on del problema}

	Se plantea hacer una traducci\'on simple del lenguaje C a Bash, de lo cual es importante destacar que se est\'a
	realizando la traducci\'on de un lenguaje fuertemente tipado a uno débilmente tipado.\\

	El objetivo principal es traducir expresiones matem\'aticas sencillas, lectura de variables, impresiones por
	pantalla y la estructura de decisión if y de repetición while. Todo esto pasando debidamente por an\'alisis 
	l\'exico, sint\'actico y sem\'antico.

\subsection{Alcance}

	Lograr traducir expresiones matem\'aticas simples, tales como: sumas, restas, multiplicaciones,
	divisiones y calculo de modulo. Para el desarrollo de dichas operaciones se acot\'o a operaciones que 
	contendran solo dos operandos. \\

	Se tendrán expresiones l\'ogicas las cuales sirven para las estructuras if y while, limitadas a un
	solo operador l\'ogico.\\

	Se leerá una sola variable por la entrada estándar y se acoto a 2 variables la cantidad que se pueden
	imprimir por pantalla.\\

	El tipo de datos en C a usar son int, float y char, empero, a pesar de reconocer el tipo de dato flotante,
	al momento de traducir solo se tomar\'a en cuenta su parte entera.

\subsection{Especificaci\'on del lenguaje fuente}

	El lenguaje fuente a ser traducido es el lenguaje C, el cual para fines pr\'acticos fue limitado a 
	operaciones matemáticas sencillas y las estructuras de decisi\'n y repetici\'on if y while, respectivamente.
	Tambi\'en son reconocidas las bibliotecas que contenga el c\'odigo fuente y la delcaraci\'on de la funci\'on
	principal "main", para esta por fines pr\'acticos no son tomados en consideraci\'on los par\'ametro
	que pueda contener (ejemplo argc o argv). \\

	Son considerados el manejos de los tipos de datos entero, carácter y flotante y mediante ellos
	implementar la estructura de asignaci\'on a variables requeridas.\\

	Para lectura de variables es utilizada la funci\'on "scanf", la cual es acotada a la lectura
	de una sola variable y la impresi\'on por panatalla mediante la funci\'on "printf", acotandola a
	una impresi\'on de un m\'aximo de 2 variables a la vez. \\

	Cualquier otra operaci\'on, comando, estructura y demás no sera reconocido por nuestra gramática 
	ni traducido a el pr\'oximo lenguaje.

\subsection{Especificaci\'on del lenguaje a traducir}

	El lenguaje de consola Bash (Bourne again shell) interpreta ordenes previamente desarrolladas en c\'odigo, y
	ejecutadas mediante un script en una shell de Unix.\\

	A ra\'iz de la traducci\'on proveniente del c\'odigo en el lenguaje C, se interpretara operaciones 
	matemáticas simples de dos operandos m\'aximo, comparación de variables, lectura y escritura de variables
	por la entrada y salida estándar, impresi\'on de un m\'aximo de 2 variables mediante pantalla 
	y caputra de valores valores mediante la misma

\section{Soluci\'on al problema planteado}

	Se plantea realizar la traducción de un lenguaje a otro s\'olo si luego de realizarse el proceso de 
	verificaci\'on de correctitud del lenguaje fuente \'este se ecuentra de manera correcta, de 
	lo contrario no se realiza y se arroja un error adecuado al problema.\\

	Para poder examinar la correctitud del c\'odigo fuente es necesario pasar a trav\'es de las siguientes 
	secciones:

\subsection{An\'alisis L\'exico}

	Encargado de detectar todos los lexemas y dirigirlos al análisis sintáctico como tokens.\\
	
	Contiene el siguiente conjunto de expresiones regulares:\\

\begin{lstlisting}

LETRA [a-zA-Z_]
DIGITO [0-9]
{LETRA}({LETRA}|{DIGITO})*
(-)?{DIGITO}+
(-)?{DIGITO}*"."{DIGITO}+

\end{lstlisting}

	Su funcionaliad radica en hacer "match" (verificar la coincidencia y correlaci\'on de tipos) para detectar 
	identificadores, tipos de datos, palabras reservadas y valores numéricos\\

	Adicionalmente detecta el siguiente conjunto de caracteres: \\

\begin{lstlisting}
. ; { } ( ) < > ! = + - * / ^ : # &
<= >= != += -= *= /= %= == || &&
++ -- /% , ' "
\end{lstlisting}

	Cada expresión regular y cada uno de los caracteres previamentes descritos envian un token al analizador 
	sintáctico, para pr\'oximamente ser utilizado.\\

	El analizador léxico también se encarga de eliminar del lenguaje fuente los espacios en blanco, saltos
	de linea y tabulaciones. Cuenta saltos de linea y rechaza aquellos caracteres que no coincidan con ningún
	patrón valido definido.


\subsection{An\'alisis Sint\'actico}

	El análisis sintáctico se realiza a través de la siguiente gramática, el cual recibe los tokens obtenidos 
	previamente y los agrupa de acuerdo a las reglas de producción de la gramática:

\begin{lstlisting}

programa:
	codigo;

codigo:
	cabecera principal | principal;

cabecera:
	cabecera NUMERAL RESERVADA MENOR ID MAYOR
	| NUMERAL RESERVADA MENOR ID MAYOR
	| cabecera NUMERAL RESERVADA COMILLAS TEXTO COMILLAS
	| NUMERAL RESERVADA COMILLAS TEXTO COMILLAS
	| cabecera NUMERAL RESERVADA MENOR ID PUNTO ID MAYOR
	| NUMERAL RESERVADA MENOR ID PUNTO ID MAYOR;

principal:
	TIPO RESERVADA PARENTESISABR PARENTESISCERR LLAVEABR cuerpo LLAVECERR;

cuerpo:
	asignacion cuerpo | asignacion | declaracion cuerpo | declaracion
	| retornar cuerpo | retornar | scan cuerpo | scan
	| print cuerpo | print | estructura cuerpo | estructura
	| estructura LLAVEABR cuerpo LLAVECERR cuerpo
	| estructura LLAVEABR cuerpo LLAVECERR
	| RESERVADA LLAVEABR cuerpo LLAVECERR cuerpo
	| RESERVADA LLAVEABR cuerpo LLAVECERR | RESERVADA cuerpo;

estructura:
	RESERVADA PARENTESISABR condicional PARENTESISCERR;

scan:
	RESERVADA PARENTESISABR COMILLAS PRCVAL COMILLAS COMA
	AMPERSAND ID PARENTESISCERR PTOCOMA;

print:
	RESERVADA PARENTESISABR COMILLAS TEXTO COMILLAS PARENTESISCERR PTOCOMA
	| RESERVADA PARENTESISABR COMILLAS TEXTO PRCVAL TEXTO COMILLAS COMA
	ID PARENTESISCERR PTOCOMA
	| RESERVADA PARENTESISABR COMILLAS PRCVAL TEXTO COMILLAS COMA
	ID PARENTESISCERR PTOCOMA
	| RESERVADA PARENTESISABR COMILLAS TEXTO PRCVAL COMILLAS COMA
	ID PARENTESISCERR PTOCOMA
	| RESERVADA PARENTESISABR COMILLAS PRCVAL COMILLAS COMA ID
	PARENTESISCERR PTOCOMA
	| RESERVADA PARENTESISABR COMILLAS PRCVAL TEXTO PRCVAL COMILLAS
	COMA ID COMA ID PARENTESISCERR PTOCOMA
	| RESERVADA PARENTESISABR COMILLAS PRCVAL TEXTO PRCVAL TEXTO PRCVAL
	COMILLAS COMA ID COMA ID COMA ID PARENTESISCERR PTOCOMA;

condicional:
	ID IGUALD ID | NUM IGUALD ID | ID IGUALD NUM
	| ID MAYOR ID | ID MAYOR_I ID | ID MENOR ID | ID MENOR_I ID
	| NUM MAYOR ID | NUM MAYOR_I ID | NUM MENOR ID | NUM MENOR_I ID
	| ID MAYOR NUM | ID MAYOR_I NUM | ID MENOR NUM | ID MENOR_I NUM;

retornar:
	RESERVADA NUM PTOCOMA | RESERVADA ID PTOCOMA
	| RESERVADA PARENTESISABR NUM PARENTESISCERR PTOCOMA
	| RESERVADA PARENTESISABR ID PARENTESISCERR PTOCOMA;


declaracion:
	TIPO ID PTOCOMA | TIPO ID IGUAL NUM PTOCOMA
	| TIPO ID IGUAL COMISIMPLE ID COMISIMPLE PTOCOMA
	| TIPO ID IGUAL COMISIMPLE NUM COMISIMPLE PTOCOMA
	| TIPO ID IGUAL ID PTOCOMA;

asignacion:
	ID IGUAL ID PTOCOMA | ID IGUAL NUM PTOCOMA | ID SUM_ASSIGN ID PTOCOMA
	| ID SUM_ASSIGN NUM PTOCOMA | ID SUB_ASSIGN ID PTOCOMA
	| ID SUB_ASSIGN NUM PTOCOMA | ID MUL_ASSIGN ID PTOCOMA
	| ID MUL_ASSIGN NUM PTOCOMA | ID IGUAL suma PTOCOMA
	| ID IGUAL resta PTOCOMA | ID IGUAL multi PTOCOMA
	| ID IGUAL div PTOCOMA | ID IGUAL ID PORCENTAJE ID PTOCOMA
	| ID IGUAL ID PORCENTAJE NUM PTOCOMA
	| ID IGUAL NUM PORCENTAJE ID PTOCOMA
	| ID IGUAL NUM PORCENTAJE NUM PTOCOMA | ID INC PTOCOMA
	| ID DEC PTOCOMA | INC ID PTOCOMA | DEC ID PTOCOMA;

suma:
	suma SUMA ID | suma SUMA NUM | ID SUMA ID | NUM SUMA NUM
	| ID SUMA NUM | NUM SUMA ID;

resta:
	resta MENOS ID | resta MENOS NUM | ID MENOS ID | NUM MENOS NUM
	| ID MENOS NUM | NUM MENOS ID;

multi:
	multi MULTI ID | multi MULTI NUM | ID MULTI ID | NUM MULTI NUM
	| ID MULTI NUM | NUM MULTI ID;

div:
	div DIV ID | div DIV NUM | ID DIV ID | NUM DIV NUM
	| ID DIV NUM | NUM DIV ID;

\end{lstlisting}

	Para la realizaci\'on del an\'alisis fue realizada la tabla de símbolos mediante un vector dinámico de 
	tuplas, el cual se encarga de agrupar el tipo de dato con su identificador.


\subsection{An\'alisis Sem\'antico}

	Este se encarga de verificar que las palabras reservadas se encuentren correctamente ubicadas dentro 
	del lenguaje fuente, que las variables esten declaradas antes de ser usadas y al momento de hacer una 
	asignación los tipos de datos  de ambas variables sean iguales.

\subsection{Manejo de errores}

	Inicia en el análisis léxico. En el se lleva cuenta de cantidad de lineas que se van leyendo con la finalidad de 
	poder indicar en que linea se ubica un error en el caso de ser detectado, ademas de emitir un mensaje si 
	no se llega a reconocer alguna palabra.\\

	En el análisis semántico ocurre una falla al momento de que no se pueda construir un árbol de 
	derivación correcto a partir de la gramática definida. Se detiene la traducción y se avisa en que linea 
	esta el error sintáctico\\

	Al momento de revisar la semántica, se valida que todas las variables sean declaradas antes de invocarse,
	al momento de ser asignadas tengan el mismo tipo de dato y por ultimo que las palabras reservadas coincidan
	con su ubicación y uso dentro del lenguaje, de lo contrario se detiene la traducción y se emite un mensaje
	de error con la linea donde se encontró y la variable o palabra reservada involucrada.

\section{Salida}

	Luego de pasar por cada una de los pasos de verificación sin levantar ninguna alerta se procede a 
	realizar la traducción según la estructura requerida del lenguaje requerido.\\
	
	En el caso particular de la traducci\'on de C a Bash toda la estructura de la cabecera en C pasa a ser una
	línea de instrucci\'on, el scanf  cambia a read, un caso similar sucede para la impresi\'on
	por pantalla,  el uso de tipos al realizar la declaraci\'on no es necesario, los condicionales dentro de 
	las estructuras de decisi\'on y repetic\'on cambian, así como el denotar el fin de las mismas, usos de 
	llaves caracteres epeciales, entre otras consideraciones que fueron tomadas para lograr que la traducci\'on 
	se realice forma correcta para el lenguaje Bash y el c\'odigo se encuentre correcto para su ejecuci\'on.\\ 
	
	Luego de ejecutar el traductor al archivo en lenguaje C, se obtiene un archivo de nombre salida.sh, el cual es 
	un script que puede ser ejecutado desde la terminal.



\end{document}
