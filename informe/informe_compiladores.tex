%%%%%%%%%%%%%%%%%%%%%%%%%%%%%%%%%%%%%%%%%
% University/School Laboratory Report
% LaTeX Template
% Version 3.1 (25/3/14)
%
% This template has been downloaded from:
% http://www.LaTeXTemplates.com
%
% Original author:
% Linux and Unix Users Group at Virginia Tech Wiki
% (https://vtluug.org/wiki/Example_LaTeX_chem_lab_report)
%
% License:
% CC BY-NC-SA 3.0 (http://creativecommons.org/licenses/by-nc-sa/3.0/)
%
%%%%%%%%%%%%%%%%%%%%%%%%%%%%%%%%%%%%%%%%%

%----------------------------------------------------------------------------------------
%	PACKAGES AND DOCUMENT CONFIGURATIONS
%----------------------------------------------------------------------------------------

%\documentclass{article}
\documentclass[aspectratio=169]{article}

\usepackage[version=3]{mhchem} % Package for chemical equation typesetting
\usepackage{siunitx} % Provides the \SI{}{} and \si{} command for typesetting SI units
\usepackage{graphicx} % Required for the inclusion of images
\usepackage{natbib} % Required to change bibliography style to APA
\usepackage{amsmath} % Required for some math elements
\usepackage[spanish]{babel}
\usepackage{hyperref}
\usepackage{float}
\usepackage{enumerate}
\usepackage{lmodern}
\usepackage[T1]{fontenc}
\usepackage{mathtools}
\usepackage[usenames,dvipsnames]{xcolor}
\usepackage{lscape}
\usepackage{pdflscape}
\usepackage{fancyvrb}
\usepackage{listings}
\usepackage{fancybox}

\setlength\parindent{0pt} % Removes all indentation from paragraphs

\renewcommand{\labelenumi}{\alph{enumi}.} % Make numbering in the enumerate environment by letter rather than number (e.g. section 6)

%\usepackage{times} % Uncomment to use the Times New Roman font

%----------------------------------------------------------------------------------------
%	DOCUMENT INFORMATION
%----------------------------------------------------------------------------------------

\title{Compiladores U-2017 \\ Informe Proyecto Traductor \\ C a Shell Script} % Title

\author{Juan Andr\'es Vivas\\
		Juli\'an Brice\~no} % Author name

\date{\today} % Date for the report

\begin{document}

\maketitle % Insert the title, author and date

\begin{center}
\begin{tabular}{l r}

\end{tabular}
\end{center}

\newpage

\section{Descripci\'on del problema}

	Se plantea hacer una traduccion simple del lenguaje C a lengaje shell, cabe destacar que estamos
	pasando de un lenguaje fuertemente tipado a uno debilmente tipado.

	Traduciremos expresiones matematicas sencillas al igual que la estructura de decision if
	y de repeticion while.


\subsection{Alcance}

	Lograr traducir expresiones matematicas simples, tales como: sumas, restas, multiplicaciones
	y divisiones, ademas de calcular el modulo.

	Expresiones logicas las cuales sirven para el if y while.

\subsection{Especificaci\'on del lenguaje fuente}

	El lenguaje fuente es C reducido, ya que solo reconoceremos las bibliotecas ( sin validacion) y
	lo que este dentro de la funcion main (sin ningun parametro). Tiene los tipos de datos entero,
	caracter y flotante.



\subsection{Especificaci\'on del lenguaje a traducir}

\section{Soluci\'on al problema planteado}

\subsection{An\'alisis L\'exico}

\subsection{An\'alisis Sint\'actico}

Gramatica:
\begin{lstlisting}

programa:
	codigo;

codigo:
	cabecera principal | principal;

cabecera:
	cabecera NUMERAL RESERVADA MENOR ID MAYOR
	| NUMERAL RESERVADA MENOR ID MAYOR
	| cabecera NUMERAL RESERVADA COMILLAS TEXTO COMILLAS
	| NUMERAL RESERVADA COMILLAS TEXTO COMILLAS
	| cabecera NUMERAL RESERVADA MENOR ID PUNTO ID MAYOR
	| NUMERAL RESERVADA MENOR ID PUNTO ID MAYOR;

principal:
	TIPO RESERVADA PARENTESISABR PARENTESISCERR LLAVEABR cuerpo LLAVECERR;

cuerpo:
	asignacion cuerpo | asignacion | declaracion cuerpo | declaracion
	| retornar cuerpo | retornar | scan cuerpo | scan 
	| print cuerpo | print | estructura cuerpo | estructura
	| RESERVADA LLAVEABR cuerpo LLAVECERR estructura PTOCOMA cuerpo
	| RESERVADA LLAVEABR cuerpo LLAVECERR estructura PTOCOMA
	| estructura LLAVEABR cuerpo LLAVECERR cuerpo
	| estructura LLAVEABR cuerpo LLAVECERR
	| RESERVADA LLAVEABR cuerpo LLAVECERR cuerpo
	| RESERVADA LLAVEABR cuerpo LLAVECERR | RESERVADA cuerpo;

estructura:
	RESERVADA PARENTESISABR condicional PARENTESISCERR;

scan:
	RESERVADA PARENTESISABR COMILLAS PRCVAL COMILLAS COMA
	AMPERSAND ID PARENTESISCERR PTOCOMA;

print:
	RESERVADA PARENTESISABR COMILLAS TEXTO COMILLAS PARENTESISCERR PTOCOMA
	| RESERVADA PARENTESISABR COMILLAS TEXTO PRCVAL TEXTO COMILLAS COMA
	ID PARENTESISCERR PTOCOMA
	| RESERVADA PARENTESISABR COMILLAS PRCVAL TEXTO COMILLAS COMA
	ID PARENTESISCERR PTOCOMA
	| RESERVADA PARENTESISABR COMILLAS TEXTO PRCVAL COMILLAS COMA
	ID PARENTESISCERR PTOCOMA
	| RESERVADA PARENTESISABR COMILLAS PRCVAL COMILLAS COMA ID
	PARENTESISCERR PTOCOMA
	| RESERVADA PARENTESISABR COMILLAS PRCVAL TEXTO PRCVAL COMILLAS
	COMA ID COMA ID PARENTESISCERR PTOCOMA
	| RESERVADA PARENTESISABR COMILLAS PRCVAL TEXTO PRCVAL TEXTO PRCVAL
	COMILLAS COMA ID COMA ID COMA ID PARENTESISCERR PTOCOMA;

condicional:
	ID IGUALD ID | NUM IGUALD ID | ID IGUALD NUM
	| ID MAYOR ID | ID MAYOR_I ID | ID MENOR ID | ID MENOR_I ID
	| NUM MAYOR ID | NUM MAYOR_I ID | NUM MENOR ID | NUM MENOR_I ID
	| ID MAYOR NUM | ID MAYOR_I NUM | ID MENOR NUM | ID MENOR_I NUM;

retornar:
	RESERVADA NUM PTOCOMA | RESERVADA ID PTOCOMA
	| RESERVADA PARENTESISABR NUM PARENTESISCERR PTOCOMA
	| RESERVADA PARENTESISABR ID PARENTESISCERR PTOCOMA;


declaracion:
	TIPO ID PTOCOMA | TIPO ID IGUAL NUM PTOCOMA
	| TIPO ID IGUAL COMISIMPLE ID COMISIMPLE PTOCOMA
	| TIPO ID IGUAL COMISIMPLE NUM COMISIMPLE PTOCOMA
	| TIPO ID IGUAL ID PTOCOMA;

asignacion:
	ID IGUAL ID PTOCOMA | ID IGUAL NUM PTOCOMA | ID SUM_ASSIGN ID PTOCOMA
	| ID SUM_ASSIGN NUM PTOCOMA | ID SUB_ASSIGN ID PTOCOMA
	| ID SUB_ASSIGN NUM PTOCOMA | ID MUL_ASSIGN ID PTOCOMA
	| ID MUL_ASSIGN NUM PTOCOMA | ID IGUAL suma PTOCOMA 
	| ID IGUAL resta PTOCOMA | ID IGUAL multi PTOCOMA 
	| ID IGUAL div PTOCOMA | ID IGUAL ID PORCENTAJE ID PTOCOMA
	| ID IGUAL ID PORCENTAJE NUM PTOCOMA 
	| ID IGUAL NUM PORCENTAJE ID PTOCOMA
	| ID IGUAL NUM PORCENTAJE NUM PTOCOMA | ID INC PTOCOMA 
	| ID DEC PTOCOMA | INC ID PTOCOMA | DEC ID PTOCOMA;

suma:
	suma SUMA ID | suma SUMA NUM | ID SUMA ID | NUM SUMA NUM
	| ID SUMA NUM | NUM SUMA ID;

resta:
	resta MENOS ID | resta MENOS NUM | ID MENOS ID | NUM MENOS NUM
	| ID MENOS NUM | NUM MENOS ID;

multi:
	multi MULTI ID | multi MULTI NUM | ID MULTI ID | NUM MULTI NUM
	| ID MULTI NUM | NUM MULTI ID;

div:
	div DIV ID | div DIV NUM | ID DIV ID | NUM DIV NUM
	| ID DIV NUM | NUM DIV ID;

\end{lstlisting}


\subsection{An\'alisis Sem\'antico}

\subsection{Manejo de errores}






\end{document}
